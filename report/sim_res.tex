\section{Simulations}\label[sec]{results}
Simulations are done in four different mission spaces with properties summed up in Table \ref{tab:worlds}.
\begin{center}
  \captionof{table}{Mission space properties\label{tab:worlds}}
  \begin{tabular}{l|l|l}
    Name & Size & Obstacles\\
    \hline
    Tinyworld & 1.5-by-1.5 square & None\\
    Tinyworld2 & 1.5-by-1.5 square & Central 0.5-by-0.5 square\\
    Rectworld & 10-by-10 square & None\\
    Complexworld & Pentagon & Horizontal wall, vertical wall and hexagon
  \end{tabular}
\end{center}
For all mission spaces simulations are performed where the dispersion term is neglected ($k_{1} = 0$), i.e.
$H(\mathbf{X}_{\mathcal{B}_{a}\cup\{a\}}) = L(\mathbf{X}_{\mathcal{B}_{a}\cup\{a\}})$, and where the dispersion term
is present ($k_{1} = 1$). In all simulations the initial configuration of the swarm is a zigzag patter starting in the bottom
left corner of the mission space moving upwards. Agents are numbered from top to bottom, meaning the top-most agent is agent $0$, and for 
a swarm of size $N$ the bottom-most agent is agent $N-1$. Thus the topmost agent will always be the first to optimize its position
when running Algorithm \ref{alg:alg1}.

Swarm configurations are plotted as follows: The boundary of visible set of an agent is plotted as a dashed line. All areas within the visible set 
of at least three agents is solid blue. Agents are visualized as orange dots.

In the step length plots, the step length trajectory of each agent is displayed as a uniquely coloured line. For simulations on large swarms
including legends would cause the figures to have to be unreasonably large. For consistency no step length plots include legends.
\clearpage
\subsection{Tinyworld}\label[secc]{tinyworld}
The Tinyworld is constructed so that for three or more agents, the entire feasible space is covered regardless of the position of the agents.

\Cref{fig:3_agnt_tw_k_1_0_distr} shows the initial and final configuration of 3 agents when spawned in the Tinyworld environment, 
and no active dispersion is used ($k_{1} = 0$).
\Cref{fig:3_agnt_tw_evolution} shows the covered area and step length per agent versus iteration count.

Results of optimizing agent positions with active dispersion ($k_{1} = 1$, $k_{2} = 1$) are shown in \Crefrange{fig:3_agnt_tw_k_1_1_k_2_1_distr}{fig:3_agnt_tw_evolution_active}.

\begin{figure}[H]
  \centering
  \includegraphics[width=\textwidth]{figs/tinyworld_3_agnt_k_1_0_k_2_1_distr.pdf}
  \caption{Initial and final configuration of 3 agents in the Tinyworld environment with $k_{1} = 0$ (no active dispersion).}
  \label{fig:3_agnt_tw_k_1_0_distr}
\end{figure}

\begin{figure}[H]
  \centering
  \begin{subfigure}[t]{0.5\textwidth}
    \centering
    \includegraphics[width=\textwidth]{figs/tinyworld_3_agnt_k_1_0_k_2_1_area_traj.pdf}
    \caption{Coverage evolution for 3 agents in the Tinyworld environment with $k_{1} = 0$ (no active dispersion).}
    \label{fig:3_agnt_tw_k_1_0_a_traj}
  \end{subfigure}%
  ~ 
  \begin{subfigure}[t]{0.5\textwidth}
    \centering
    \includegraphics[width=\textwidth]{figs/tinyworld_3_agnt_k_1_0_k_2_1_step_traj.pdf}
    \caption{Step length evolution for 3 agents in the Tinyworld environment with $k_{1} = 0$ (no active dispersion).}
    \label{fig:3_agnt_tw_k_1_0_s_traj}
  \end{subfigure}
  \caption{Coverage percentage and step length evolution for 3 agents in the Tinyworld environment when no active dispersion is used.}
  \label{fig:3_agnt_tw_evolution}
\end{figure}

\begin{figure}[H]
  \centering
  \includegraphics[width=\textwidth]{figs/tinyworld_3_agnt_k_1_1_k_2_1_distr.pdf}
  \caption{Initial and final configuration of 3 agents in the Tinyworld environment with $k_{1} = k_{2} = 1$ (active dispersion).}
  \label{fig:3_agnt_tw_k_1_1_k_2_1_distr}
\end{figure}

\begin{figure}[H]
  \centering
  \begin{subfigure}[t]{0.5\textwidth}
    \centering
    \includegraphics[width=\textwidth]{figs/tinyworld_3_agnt_k_1_1_k_2_1_area_traj.pdf}
    \caption{Coverage evolution for 3 agents in the Tinyworld environment with $k_{1} = k_{2} = 1$ (active dispersion).}
    \label{fig:3_agnt_tw_k_1_k_2_1_a_traj}
  \end{subfigure}%
  ~ 
  \begin{subfigure}[t]{0.5\textwidth}
    \centering
    \includegraphics[width=\textwidth]{figs/tinyworld_3_agnt_k_1_1_k_2_1_step_traj.pdf}
    \caption{Step length evolution for 3 agents in the Tinyworld environment with $k_{1} = k_{2} = 1$ (active dispersion).}
    \label{fig:3_agnt_tw_k_1_k_2_1_s_traj}
  \end{subfigure}
  \caption{Coverage percentage and step length evolution for 3 agents in the Tinyworld environment when active dispersion is used.}
  \label{fig:3_agnt_tw_evolution_active}
\end{figure}
\clearpage
\subsection{Tinyworld2}\label[secc]{tinyworld2}
The Tinyworld2 environment is constructed to study how obstacles affect the configuration generated by Algorithm \ref{alg:alg1}.

The results of a simulation for 3 agents without active dispersion ($k_{1} = 0$) is shown in \Crefrange{fig:3_agnt_tw2_k_1_0_distr}{fig:3_agnt_tw2_evolution}.
\Crefrange{fig:6_agnt_tw2_k_1_0_distr}{fig:6_agnt_tw2_evolution} show the results of a simulation of 6 agents without active dispersion.

The results generated by Algorithm \ref{alg:alg1} when applying active dispersion ($k_{1} = 1$, $k_{1} = 2$) and running simulations for 3 and 6 agents 
are shown in \Crefrange{fig:3_agnt_tw2_k_1_1_k_2_1_distr}{fig:3_agnt_tw2_evolution_active} and \Crefrange{fig:6_agnt_tw2_k_1_1_k_2_1_distr}{fig:6_agnt_tw2_evolution_active} respectively.
\begin{figure}[H]
  \centering
  \includegraphics[width=\textwidth]{figs/tinyworld2_3_agnt_k_1_0_k_2_1_distr.pdf}
  \caption{Inital and final configuration of 3 agents in the Tinyworld2 environment with $k_{1} = 0$ (no active dispersion).}
  \label{fig:3_agnt_tw2_k_1_0_distr}
\end{figure}
\begin{figure}[H]
  \centering
  \begin{subfigure}[t]{0.5\textwidth}
    \centering
    \includegraphics[width=\textwidth]{figs/tinyworld2_3_agnt_k_1_0_k_2_1_area_traj.pdf}
    \caption{Coverage evolution for 3 agents in the Tinyworld2 environment with $k_{1} = 0$ (no active dispersion).}
    \label{fig:3_agnt_tw2_k_1_0_a_traj}
  \end{subfigure}%
  ~ 
  \begin{subfigure}[t]{0.5\textwidth}
    \centering
    \includegraphics[width=\textwidth]{figs/tinyworld2_3_agnt_k_1_0_k_2_1_step_traj.pdf}
    \caption{Step length evolution for 3 agents in the Tinyworld2 environment with $k_{1} = 0$ (no active dispersion).}
    \label{fig:3_agnt_tw2_k_1_0_s_traj}
  \end{subfigure}
  \caption{Coverage percentage and step length evolution for 3 agents in the Tinyworld2 environment when no active dispersion is used.}
  \label{fig:3_agnt_tw2_evolution}
\end{figure}


\begin{figure}[H]
  \centering
  \includegraphics[width=\textwidth]{figs/tinyworld2_3_agnt_k_1_1_k_2_1_distr.pdf}
  \caption{Inital and final configuration of 3 agents in the Tinyworld2 environment with $k_{1} = k_{2} = 1$ (active dispersion).}
  \label{fig:3_agnt_tw2_k_1_1_k_2_1_distr}
\end{figure}
\begin{figure}[H]
  \centering
  \begin{subfigure}[t]{0.5\textwidth}
    \centering
    \includegraphics[width=\textwidth]{figs/tinyworld2_3_agnt_k_1_1_k_2_1_area_traj.pdf}
    \caption{Coverage evolution for 3 agents in the Tinyworld2 environment with $k_{1} = k_{2} = 1$ (active dispersion).}
    \label{fig:3_agnt_tw2_k_1_1_a_traj}
  \end{subfigure}%
  ~ 
  \begin{subfigure}[t]{0.5\textwidth}
    \centering
    \includegraphics[width=\textwidth]{figs/tinyworld2_3_agnt_k_1_1_k_2_1_step_traj.pdf}
    \caption{Step length evolution for 3 agents in the Tinyworld2 environment with $k_{1} = k_{2} = 1$ (active dispersion).}
    \label{fig:3_agnt_tw2_k_1_1_s_traj}
  \end{subfigure}
  \caption{Coverage percentage and step length evolution for 3 agents in the Tinyworld2 environment when active dispersion is used.}
  \label{fig:3_agnt_tw2_evolution_active}
\end{figure}

\begin{figure}[H]
  \centering
  \includegraphics[width=\textwidth]{figs/tinyworld2_6_agnt_k_1_0_k_2_1_distr.pdf}
  \caption{Inital and final configuration of 6 agents in the Tinyworld2 environment with $k_{1} = 0$ (no active dispersion).}
  \label{fig:6_agnt_tw2_k_1_0_distr}
\end{figure}
\begin{figure}[H]
  \centering
  \begin{subfigure}[t]{0.5\textwidth}
    \centering
    \includegraphics[width=\textwidth]{figs/tinyworld2_6_agnt_k_1_0_k_2_1_area_traj.pdf}
    \caption{Coverage evolution for 6 agents in the Tinyworld2 environment with $k_{1} = 0$ (no active dispersion).}
    \label{fig:6_agnt_tw2_k_1_0_a_traj}
  \end{subfigure}%
  ~ 
  \begin{subfigure}[t]{0.5\textwidth}
    \centering
    \includegraphics[width=\textwidth]{figs/tinyworld2_6_agnt_k_1_0_k_2_1_step_traj.pdf}
    \caption{Step length evolution for 6 agents in the Tinyworld2 environment with $k_{1} = 0$ (no active dispersion).}
    \label{fig:6_agnt_tw2_k_1_0_s_traj}
  \end{subfigure}
  \caption{Coverage percentage and step length evolution for 6 agents in the Tinyworld2 environment when active dispersion is not applied.}
  \label{fig:6_agnt_tw2_evolution}
\end{figure}

\begin{figure}[H]
  \centering
  \includegraphics[width=\textwidth]{figs/tinyworld2_6_agnt_k_1_1_k_2_1_distr.pdf}
  \caption{Inital and final configuration of 6 agents in the Tinyworld2 environment with $k_{1} = k_{2} = 1$ (active dispersion).}
  \label{fig:6_agnt_tw2_k_1_1_k_2_1_distr}
\end{figure}
\begin{figure}[H]
  \centering
  \begin{subfigure}[t]{0.5\textwidth}
    \centering
    \includegraphics[width=\textwidth]{figs/tinyworld2_6_agnt_k_1_1_k_2_1_area_traj.pdf}
    \caption{Coverage evolution for 6 agents in the Tinyworld2 environment with $k_{1} = k_{1} = 1$ (active dispersion).}
    \label{fig:6_agnt_tw2_k_1_1_a_traj}
  \end{subfigure}%
  ~ 
  \begin{subfigure}[t]{0.5\textwidth}
    \centering
    \includegraphics[width=\textwidth]{figs/tinyworld2_6_agnt_k_1_1_k_2_1_step_traj.pdf}
    \caption{Step length evolution for 6 agents in the Tinyworld2 environment with $k_{1} = k_{2} = 1$ (active dispersion).}
    \label{fig:6_agnt_tw2_k_1_1_s_traj}
  \end{subfigure}
  \caption{Coverage percentage and step length evolution for 6 agents in the Tinyworld2 environment when active dispersion is applied.}
  \label{fig:6_agnt_tw2_evolution_active}
\end{figure}

\clearpage
\subsection{Rectworld}\label[secc]{rectworld}
The Rectworld is constructed to examine the behaviour of Algorithm \ref{alg:alg1} for a larger mission space. Simulations are performed
for swarms of 3, 6 and 15 agents.

%\Crefrange{fig:3_agnt_bw_k_1_0_k_2_1_distr}{fig:3_agnt_bw_evolution}, \Crefrange{fig:6_agnt_bw_k_1_0_k_2_1_distr}{fig:6_agnt_bw_evolution}
%and \Crefrange{fig:15_agnt_bw_k_1_0_k_2_1_distr}{fig:15_agnt_bw_evolution} show the results of simulating without active dispersion ($k_{1} = 0$) for 3, 6 and 15
%agents respectively.
%
%\Crefrange{fig:3_agnt_bw_k_1_1_k_2_1_distr}{fig:3_agnt_bw_evolution_active}, \Crefrange{fig:6_agnt_bw_k_1_1_k_2_1_distr}{fig:6_agnt_bw_evolution_active}
%and \Crefrange{fig:15_agnt_bw_k_1_1_k_2_1_distr}{fig:15_agnt_bw_evolution_active} show the results of simulating with active dispersion ($k_{1} = k_{2} = 1$) for 3, 6 and 15
%agents respectively.

\begin{figure}[H]
  \centering
  \includegraphics[width=\textwidth]{figs/bigworld_3_agnt_k_1_0_k_2_1_distr.pdf}
  \caption{Inital and final configuration of 3 agents in the Bigworld environment with $k_{1} = 0$ (no active dispersion).}
  \label{fig:3_agnt_bw_k_1_0_k_2_1_distr}
\end{figure}
\begin{figure}[H]
  \centering
  \begin{subfigure}[t]{0.5\textwidth}
    \centering
    \includegraphics[width=\textwidth]{figs/bigworld_3_agnt_k_1_0_k_2_1_area_traj.pdf}
    \caption{Coverage evolution for 3 agents in the Bigworld environment with $k_{1} = 0$ (no active dispersion).}
    \label{fig:3_agnt_bw_k_1_0_a_traj}
  \end{subfigure}%
  ~ 
  \begin{subfigure}[t]{0.5\textwidth}
    \centering
    \includegraphics[width=\textwidth]{figs/bigworld_3_agnt_k_1_0_k_2_1_step_traj.pdf}
    \caption{Step length evolution for 3 agents in the Bigworld environment with $k_{1} = 0$ (no active dispersion).}
    \label{fig:3_agnt_bw_k_1_0_s_traj}
  \end{subfigure}
  \caption{Coverage percentage and step length evolution for 3 agents in the Bigworld environment when no active dispersion is used.}
  \label{fig:3_agnt_bw_evolution}
\end{figure}


\begin{figure}[H]
  \centering
  \includegraphics[width=\textwidth]{figs/bigworld_3_agnt_k_1_1_k_2_1_distr.pdf}
  \caption{Inital and final configuration of 3 agents in the Bigworld environment with $k_{1} = k_{2} = 1$ (active dispersion).}
  \label{fig:3_agnt_bw_k_1_1_k_2_1_distr}
\end{figure}
\begin{figure}[H]
  \centering
  \begin{subfigure}[t]{0.5\textwidth}
    \centering
    \includegraphics[width=\textwidth]{figs/bigworld_3_agnt_k_1_1_k_2_1_area_traj.pdf}
    \caption{Coverage evolution for 3 agents in the Bigworld environment with $k_{1} = k_{1} = 1$ (active dispersion).}
    \label{fig:3_agnt_bw_k_1_1_a_traj}
  \end{subfigure}%
  ~ 
  \begin{subfigure}[t]{0.5\textwidth}
    \centering
    \includegraphics[width=\textwidth]{figs/bigworld_3_agnt_k_1_1_k_2_1_step_traj.pdf}
    \caption{Step length evolution for 3 agents in the Bigworld environment with $k_{1} = k_{1} = 1$ (active dispersion).}
    \label{fig:3_agnt_bw_k_1_1_s_traj}
  \end{subfigure}
  \caption{Coverage percentage and step length evolution for 3 agents in the Bigworld environment when active dispersion is used.}
  \label{fig:3_agnt_bw_evolution_active}
\end{figure}

%%%

\begin{figure}[H]
  \centering
  \includegraphics[width=\textwidth]{figs/bigworld_6_agnt_k_1_0_k_2_1_distr.pdf}
  \caption{Inital and final configuration of 6 agents in the Bigworld environment with $k_{1} = 0$ (no active dispersion).}
  \label{fig:6_agnt_bw_k_1_0_k_2_1_distr}
\end{figure}
\begin{figure}[H]
  \centering
  \begin{subfigure}[t]{0.5\textwidth}
    \centering
    \includegraphics[width=\textwidth]{figs/bigworld_6_agnt_k_1_0_k_2_1_area_traj.pdf}
    \caption{Coverage evolution for 6 agents in the Bigworld environment with $k_{1} = 0$ (no active dispersion).}
    \label{fig:6_agnt_bw_k_1_0_a_traj}
  \end{subfigure}%
  ~ 
  \begin{subfigure}[t]{0.5\textwidth}
    \centering
    \includegraphics[width=\textwidth]{figs/bigworld_6_agnt_k_1_0_k_2_1_step_traj.pdf}
    \caption{Step length evolution for 6 agents in the Bigworld environment with $k_{1} = 0$ (no active dispersion).}
    \label{fig:6_agnt_bw_k_1_0_s_traj}
  \end{subfigure}
  \caption{Coverage percentage and step length evolution for 6 agents in the Bigworld environment when no active dispersion is used.}
  \label{fig:6_agnt_bw_evolution}
\end{figure}


\begin{figure}[H]
  \centering
  \includegraphics[width=\textwidth]{figs/bigworld_6_agnt_k_1_1_k_2_1_distr.pdf}
  \caption{Inital and final configuration of 6 agents in the Bigworld environment with $k_{1} = k_{2} = 1$ (active dispersion).}
  \label{fig:6_agnt_bw_k_1_1_k_2_1_distr}
\end{figure}
\begin{figure}[H]
  \centering
  \begin{subfigure}[t]{0.5\textwidth}
    \centering
    \includegraphics[width=\textwidth]{figs/bigworld_6_agnt_k_1_1_k_2_1_area_traj.pdf}
    \caption{Coverage evolution for 6 agents in the Bigworld environment with $k_{1} = k_{1} = 1$ (active dispersion).}
    \label{fig:6_agnt_bw_k_1_1_a_traj}
  \end{subfigure}%
  ~ 
  \begin{subfigure}[t]{0.5\textwidth}
    \centering
    \includegraphics[width=\textwidth]{figs/bigworld_6_agnt_k_1_1_k_2_1_step_traj.pdf}
    \caption{Step length evolution for 6 agents in the Bigworld environment with $k_{1} = k_{1} = 1$ (active dispersion).}
    \label{fig:6_agnt_bw_k_1_1_s_traj}
  \end{subfigure}
  \caption{Coverage percentage and step length evolution for 6 agents in the Bigworld environment when active dispersion is used.}
  \label{fig:6_agnt_bw_evolution_active}
\end{figure}

%%%

\begin{figure}[H]
  \centering
  \includegraphics[width=\textwidth]{figs/bigworld_15_agnt_k_1_0_k_2_1_distr.pdf}
  \caption{Inital and final configuration of 15 agents in the Bigworld environment with $k_{1} = 0$ (no active dispersion).}
  \label{fig:15_agnt_bw_k_1_0_k_2_1_distr}
\end{figure}
\begin{figure}[H]
  \centering
  \begin{subfigure}[t]{0.5\textwidth}
    \centering
    \includegraphics[width=\textwidth]{figs/bigworld_15_agnt_k_1_0_k_2_1_area_traj.pdf}
    \caption{Coverage evolution for 15 agents in the Bigworld environment with $k_{1} = 0$ (no active dispersion).}
    \label{fig:15_agnt_bw_k_1_0_a_traj}
  \end{subfigure}%
  ~ 
  \begin{subfigure}[t]{0.5\textwidth}
    \centering
    \includegraphics[width=\textwidth]{figs/bigworld_15_agnt_k_1_0_k_2_1_step_traj.pdf}
    \caption{Step length evolution for 15 agents in the Bigworld environment with $k_{1} = 0$ (no active dispersion).}
    \label{fig:15_agnt_bw_k_1_0_s_traj}
  \end{subfigure}
  \caption{Coverage percentage and step length evolution for 15 agents in the Bigworld environment when no active dispersion is used.}
  \label{fig:15_agnt_bw_evolution}
\end{figure}


\begin{figure}[H]
  \centering
  \includegraphics[width=\textwidth]{figs/bigworld_15_agnt_k_1_1_k_2_1_distr.pdf}
  \caption{Inital and final configuration of 15 agents in the Bigworld environment with $k_{1} = k_{2} = 1$ (active dispersion).}
  \label{fig:15_agnt_bw_k_1_1_k_2_1_distr}
\end{figure}
\begin{figure}[H]
  \centering
  \begin{subfigure}[t]{0.5\textwidth}
    \centering
    \includegraphics[width=\textwidth]{figs/bigworld_15_agnt_k_1_1_k_2_1_area_traj.pdf}
    \caption{Coverage evolution for 15 agents in the Bigworld environment with $k_{1} = k_{1} = 1$ (active dispersion).}
    \label{fig:15_agnt_bw_k_1_1_a_traj}
  \end{subfigure}%
  ~ 
  \begin{subfigure}[t]{0.5\textwidth}
    \centering
    \includegraphics[width=\textwidth]{figs/bigworld_15_agnt_k_1_1_k_2_1_step_traj.pdf}
    \caption{Step length evolution for 15 agents in the Bigworld environment with $k_{1} = k_{1} = 1$ (active dispersion).}
    \label{fig:15_agnt_bw_k_1_1_s_traj}
  \end{subfigure}
  \caption{Coverage percentage and step length evolution for 15 agents in the Bigworld environment when active dispersion is used.}
  \label{fig:15_agnt_bw_evolution_active}
\end{figure}
\clearpage
\subsection{Complexworld}\label[secc]{complexworld}
The Complexworld is constructed to evaluate the performance of Algorithm \ref{alg:alg1} for a large swarm in a larger and more demanding environment.
Simulations are performed with and without active dispersion for a swarm of 50 agents. The results are shown in 
\Crefrange{fig:50_agnt_cw_k_1_0_k_2_1_distr}{fig:50_agnt_tw_evolution_active}.
\begin{figure}[H]
  \centering
  \includegraphics[width=\textwidth]{figs/complexworld_50_agnt_k_1_0_k_2_1_distr.pdf}
  \caption{Initial and final configuration of 50 agents in the Complexworld environment with $k_{1} = k_{2} = 1$ (active dispersion).}
  \label{fig:50_agnt_cw_k_1_0_k_2_1_distr}
\end{figure}

\begin{figure}[H]
  \centering
  \begin{subfigure}[t]{0.5\textwidth}
    \centering
    \includegraphics[width=\textwidth]{figs/complexworld_50_agnt_k_1_0_k_2_1_area_traj.pdf}
    \caption{Coverage evolution for 50 agents in the Complexworld environment with $k_{1} = k_{2} = 1$ (active dispersion).}
    \label{fig:50_agnt_cw_k_1_0_k_2_1_a_traj}
  \end{subfigure}%
  ~ 
  \begin{subfigure}[t]{0.5\textwidth}
    \centering
    \includegraphics[width=\textwidth]{figs/complexworld_50_agnt_k_1_0_k_2_1_step_traj.pdf}
    \caption{Step length evolution for 50 agents in the Complexworld environment with $k_{1} = k_{2} = 1$ (active dispersion).}
    \label{fig:50_agnt_cw_k_1_0_k_2_1_s_traj}
  \end{subfigure}
  \caption{Coverage percentage and step length evolution for 50 agents in the Complexworld environment when active dispersion is used.}
  \label{fig:50_agnt_cw_evolution}
\end{figure}


\begin{figure}[H]
  \centering
  \includegraphics[width=\textwidth]{figs/complexworld_50_agnt_k_1_1_k_2_1_distr.pdf}
  \caption{Initial and final configuration of 50 agents in the Complexworld environment with $k_{1} = k_{2} = 1$ (active dispersion).}
  \label{fig:50_agnt_cw_k_1_1_k_2_1_distr}
\end{figure}

\begin{figure}[H]
  \centering
  \begin{subfigure}[t]{0.5\textwidth}
    \centering
    \includegraphics[width=\textwidth]{figs/complexworld_50_agnt_k_1_1_k_2_1_area_traj.pdf}
    \caption{Coverage evolution for 50 agents in the Complexworld environment with $k_{1} = k_{2} = 1$ (active dispersion).}
    \label{fig:50_agnt_cw_k_1_k_2_1_a_traj}
  \end{subfigure}%
  ~ 
  \begin{subfigure}[t]{0.5\textwidth}
    \centering
    \includegraphics[width=\textwidth]{figs/complexworld_50_agnt_k_1_1_k_2_1_step_traj.pdf}
    \caption{Step length evolution for 50 agents in the Complexworld environment with $k_{1} = k_{2} = 1$ (active dispersion).}
    \label{fig:50_agnt_cw_k_1_1_k_2_1_s_traj}
  \end{subfigure}
  \caption{Coverage percentage and step length evolution for 50 agents in the Complexworld environment when active dispersion is used.}
  \label{fig:50_agnt_tw_evolution_active}
\end{figure}
