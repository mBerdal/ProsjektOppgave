\section{Implementation}
\subsection{Local probability}
It is assumed that the agents have perfect communication capabilities within their maximum range. The local probability of an agent $a$ wrt. a point $\mathbf{y}$ is defined as:
\begin{equation}
  \hat{p}(\mathbf{x}_{a}, \mathbf{y}) = \begin{cases}
    1, &\mathbf{y}\in V_{a}\\
    0, &\mathbf{y}\in U_{a}
  \end{cases} = 1_{\big\{\mathbf{y}\in V_{a}\big\}},
\end{equation}
where $1_{\{\cdot\}}$ is the indicator function, which is simply equal to one if the clause in the subscript is true and zero otherwise.
This implies that the overall probability of coverage over the feasible space in \eqref{rewritten_objective} is simply the area of all intersections of three or more visible sets.

\subsection{Computing the local probability of coverage}
The neighbors of an agent $a$ is partitioned into two sets:
\begin{subequations}
  \begin{equation}
    \mathcal{B}_{a V} = \{j\in\mathcal{B}_{a}: \mathbf{y}\in V_{j}\}
  \end{equation}
  \begin{equation}
    \mathcal{B}_{a U} = \{j\in\mathcal{B}_{a}: \mathbf{y}\in U_{j}\}.
  \end{equation}
\end{subequations}
Thus for a given point $\mathbf{y}$, the set $\mathcal{B}_{a V}$ contains all neighbors of $a$ whose visible set contains $\mathbf{y}$,
and $\mathcal{B}_{a U}$ contains all neighbors of agent $a$ whose visible set does not contain $\mathbf{y}$.

Now the local probability of coverage for agent $a$ can be written as:
\begin{equation}\label[eq]{loc_prob_cov_implt}
  \begin{split}
    L(\mathbf{X}_{\mathcal{B}_{a}\cup\{a\}}) &= \int_{V_{a}}\Phi^{2}(\mathbf{X}_{\mathcal{B}_{a}}, \mathbf{y})1_{\{\mathbf{y}\in V_{a}\}}d\mathbf{y} = \int_{V_{a}}\Phi^{2}(\mathbf{X}_{\mathcal{B}_{a}}, \mathbf{y})d\mathbf{y}\\
    &= \int_{V_{a}}\sum_{n = 0}^{2} \Phi^{n}(\mathbf{X}_{\mathcal{B}_{a V}}, \mathbf{y})\Phi^{2-n}(\mathbf{X}_{\mathcal{B}_{a U}},\mathbf{y})d\mathbf{y}\\
    &= \int_{V_{a}}\sum_{n = 0}^{2}1_{\{|B_{a V}| = n\}}1_{\{2-n = 0\}}d\mathbf{y}\\
    &= \int_{V_{a}}1_{\{|B_{a V}| = 2\}}d\mathbf{y}.
  \end{split}
\end{equation}
Thus the value of the local probability of coverage is equal to the area where the visible set of $a$ overlaps with those of exactly two neighboring agents when assuming perfect communication within the entire visible set.

An alternative way of computing the value of the local probability of coverage in \eqref{loc_prob_cov_implt}, used in the implementation,
is presented below:
\begin{equation}\label[eq]{two_neigh_area}
  \begin{split}
    L(\mathbf{X}_{\mathcal{B}_{a}\cup\{a\}}) &= A\Biggr(\bigcup_{\mathcal{A}\in Comb(\mathcal{B}_{a}, 2)}\Bigg[\bigg(V_{a}\cap \bigcap_{i\in\mathcal{A}}V_{i}\bigg)\setminus\bigg(\bigcup_{j\in\mathcal{B}_{a}\setminus\mathcal{A}}V_{j}\bigg)\Bigg]\Biggr)\\
    &= A\Biggr(V_{a}\cap\bigcup_{\mathcal{A}\in Comb(\mathcal{B}_{a}, 2)}\Bigg[\bigg(\bigcap_{i\in\mathcal{A}}V_{i}\bigg)\setminus\bigg(\bigcup_{j\in\mathcal{B}_{a}\setminus\mathcal{A}}V_{j}\bigg)\Bigg]\Biggr),
  \end{split}
\end{equation}
where $A(\cdot)$ returns the area of its argument.
\subsection{Optimizing swarm configuration}
Running simulations is done by iteratively solving \eqref{local_opt_prob} for one agent at a time. Meaning at any time there is only
a single agent computing its optimal position while all other agents are static. Once the agent currently computing its optimal position
has done so, it moves to the optimum. Then the next agent does the same, and so on. This procedure is repeated
until all agents are sufficiently close to a local optimum. The procedure is described in Algorithm \ref{alg:alg1}.

\begin{algorithm}[H]
  \SetAlgoLined
  \KwIn{Size of swarm: $N$, Initial configuration of swarm: $\mathbf{X}_{0}$, Feasible space: $\mathcal{F}$, Tolerance: $\epsilon$}

  $\mathbf{X}_{\mathcal{N}}\gets \mathbf{X}_{0}$\;
  \For{$a\gets 0$ \KwTo $N-1$}{
    $\mathbf{C}[a]\gets\mathrm{False}$\;
    $\mathbf{V}[a]\gets$ Compute $V_{a}$ using \eqref{visible_set_def} with $\mathbf{x}_{a} = \mathbf{X}_{\mathcal{N}}[a]$\;
  }
   \While{not $\mathbf{C}[a]$ is $\mathrm{True}$ for all $a = 0\hdots N-1$}{
     \For{$a\gets 0$ \KwTo $N-1$}{
       $\mathbf{x}_{a, 0}\gets\mathbf{X}_{\mathcal{N}}[a]$\;
       $\mathcal{B}_{a}\gets$ Compute $\mathcal{B}_{a}$ using \eqref{neigh_def} with $\mathbf{x}_{a} = \mathbf{x}_{a, 0}$\;
       $\mathbf{x}_{a}^{*}\gets$ Solve \eqref{local_opt_prob} with initial guess $\mathbf{x}_{a, 0}$ and neighbors $\mathcal{B}_{a}$\;
       \If{$\norm{\mathbf{x}_{a}^{*} - \mathbf{x}_{a, 0}}\leq\epsilon$}{
         $\mathbf{C}[a]\gets\mathrm{True}$
       }
       $\mathbf{V}[a]\gets$ Compute $V_{a}$ using \eqref{visible_set_def} with $\mathbf{x}_{a} = \mathbf{x}_{a}^{*}$\;
       $\mathbf{X}_{\mathcal{N}}[a]\gets\mathbf{x}_{a}^{*}$\;
     }
   }
   \KwRet{$\mathbf{X}_{\mathcal{N}}$}
   \caption{Optimizing swarm configuration}
   \label{alg:alg1}
  \end{algorithm}
\subsection{Optimization solver}
As \eqref{local_opt_prob} is a non-convex and constrained optimization problem it is solved using Sequential Quadratic Programming (SQP). 
In \cite{kraft1988software} a Nonlinear Program (NLP) is defined as:
\begin{equation}\label[eq]{nlp}
  \begin{split}
    \min_{\mathbf{x}\in\mathbb{R}^{n}}f(\mathbf{x})\quad\mathrm{s.t.}\quad&g_{i}(\mathbf{x}) = 0\;i\in\{1\hdots m_{e}\}\\
    &g_{i}(\mathbf{x}) \geq 0\;i\in\{m_{e}+1\hdots m\}\\
    &\mathbf{x}^{lb}\leq\mathbf{x}\leq\mathbf{x}^{ub}
  \end{split}
\end{equation}
where $f:\mathbb{R}^{n}\rightarrow\mathbb{R}$ and $g:\mathbb{R}^{n}\rightarrow\mathbb{R}^{m}$ are assumed to be continuously differentiable. A SQP (Sequential Quadratic Programming) solver finds one, 
of possibly many, local minima of $f(\mathbf{x})$ within the feasible space defined by the constraints. The solution
of \eqref{nlp} is found in an iterative manner. Given an initial guess, $\mathbf{x}_{0}$, of the solution, a SQP solver iteratively solves quadratic sub-problems, generating step size $\alpha_{k}$ and search 
direction $\mathbf{d}_{k}$. The solution to \eqref{nlp} is then updated as:
\begin{equation}
  \mathbf{x}_{k+1} = \mathbf{x}_{k} + \alpha_{k}\mathbf{d}_{k}
\end{equation}
Iteration is performed until some optimality condition is fulfilled \cite{kraft1988software}.

Algorithm \ref{alg:alg1} is implemented in the Python programming language \cite{python}. The Scipy \cite{2020SciPy-NMeth} optimization library implements a wrapper
for the SQP subroutine proposed in \cite{kraft1988software} which is used for solving \eqref{local_opt_prob}.
\subsection{Parameters}
In all simulations the following parameter values are chosen
\begin{center}
  \begin{tabular}{l|c|c}
     & Variable & Value\\
    \hline
    Maximum communication range & $r$ & $3$\\
    Minimum distance to neighbors & $r_{min}$ & $0.2$\\
    Minimum distance to line connecting neighbors & $d_{min}$ & $0.1$\\
    Convergence threshold & $\epsilon$ & $10^{-2}$\\
    Dispersion decay & $k_{2}$ & $1$\\
    Dispersion gain & $k_{1}$ & $\{0, 1\}$
  \end{tabular}
\end{center}
\subsection{Source code}
The source code implementing Algorithm \ref{alg:alg1} and producing all plots shown in this report is 
accessible at \cite{repo}.