\begin{abstract}
First responders (FRs) in search-and-rescue missions frequently expose themselves to unknown and dangerous environments. With the increasing 
capabilities of micro areal vehicles (MAVs) and other mobile agent technologies, some of the risks FRs must take in order to save lives and secure the 
well-being of others can be mitigated. Deploying a swarm of MAVs into the environment to facilitate localization of the FRs and others
can vastly improve the security of all parties involved in search-and-rescue missions.

In this report, multi-agent networks, and how these can be deployed into known areas to provide an infrastructure 
for localization is studied. A novel optimization-based distributed coverage control algorithm for deployment of a multi-agent multilateration network is proposed.
The distributed objective function consists of two parts: local probability of coverage and dispersion from neighbors.

Simulations are performed in different environments and for varying swarm sizes. For all environments and swarm sizes simulations are performed
where agents disregard dispersion from neighbors, and where agents actively disperse from their neighbors. The results obtained through 
simulations show that the initial coverage is always improved upon, when theoretically possible, by the proposed algorithm when agents disregard dispersing from their neighbors.

Simulations also show that actively dispersing agents, in some cases, can prevent the swarm from settling at clearly sub-optimal configurations where
the environment is not, but could be, further explored and thus yield a substantially larger covered area. In other cases simulations show
that actively dispersing agents cause a shift in behavior where agents reach a configuration in which maximum dispersion is reached,
rather than the maximum covered area.

The results presented in this report show that the proposed coverage control algorithm cannot yet be utilized in real-world situations as it lacks robustness. 
However, this report lays the groundwork for further research and development. In an upcoming master's thesis,
the multilateration network deployment problem will be further examined. This will hopefully yield results that can directly reduce the dangers 
to which FRs expose themselves.
\newline\newline
This project has been carried out in collaboration with the INGENIOUS project funded by the European Union's Horizon 2020 research and 
innovation program under grant agreement No 833435 and the Norwegian Research Council project "Autonomous Underwater Fleets" under the grant agreement No 302435.

\end{abstract}