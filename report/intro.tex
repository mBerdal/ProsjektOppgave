\section{Introduction}

A first responder (FR) is a person who is among those responsible for going immediately to the scene of an accident or emergency to provide assistance \cite{FR}.
They will typically be employed by the emergency services such as the police, fire department or health services, and take it upon themselves to secure the health of people, property and the environment.
Often this includes sacrificing their own safety in order to secure that of others. The need for quick action gives them no other choice.

Search-and-rescue personnel in particular expose themselves to dangerous environments. Burning buildings, collapsed or flooded caves etc. are just some examples of the rapid-changing
hazardous environments in which FRs can find themselves. Often they enter without any knowledge of what is waiting for them on the other side. Imagine a burning building. The local fire department has just arrived. There is no time to 
spare, so the first firefighters enter to save the lives of those inside before the floor plan is inspected. It might not even be valid as the roaring fire continues to eat up the walls and support beams for the roof. Still the firefighters enter in order to rescue those inside.
If something is to happen to the entering firefighters there is no way to locate them, and the ones entering to save them still don't know what is waiting for them inside.

Advancements in technology can be used to limit the dangers to which FRs expose themselves. With the increasing performance and decreasing cost of mobile robots \cite{MAV_enabling}, and the ability to outfit them with whatever equipment one might think of, it is relevant to think of how 
this might be used to the advantage of first responders. Using disposable robots for tasks such as mapping unknown environments could drastically improve the safety of FRs. In the example stated above, mobile robots could be dispatched into the burning house
and continuously feed the firefighters with a real time map of the environment. Robots could also enter the building to set up an ad-hoc network, a dynamic and self-configuring network formed by a collection of mobile nodes \cite{GAVHALE2016477}, used for locating FRs inside GNSS denied environments, 
or supply information about events happening, such as changes in the environment or the presence of poisonous gasses etc.

The INGENIOUS project is a EU-funded project that has as its mission to develop and test a next generation integrated toolkit (NGIT) for collaborative response. This includes 
developing micro indoor aerial drones that are to enter dangerous buildings in order to create an indoor mesh network that allows for indoor localization of FRs.
Developing a distributed deployment scheme in order to provide such a localization network is the task of this report.