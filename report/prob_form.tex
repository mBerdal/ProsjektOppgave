\section{Problem formulation}
Using a swarm, $\mathcal{N}$, of $N$ mobile agents with homogenous communication range, the swarm should be deployed in a mission space, $\Omega$. 
The agents should, without centralized control, spread throughout the mission space and position themselves 
such that they can be used as beacons in a multilateration scheme 
in order to deliver precise positional data to entities entering the mission space.

\subsection{Coverage}\label{secc:coverage}
It is clear from subsection \ref{trilat} that in two-dimensional space three or more agents are needed to perform the task of multilateration. Hence a point $\mathbf{y}$ is said
to be \textit{covered} iff. the local probability of at least three agents with respect to $\mathbf{y}$ is non-zero. Thus, given a swarm $\mathcal{N}$, the point $\mathbf{y}$ is covered
iff.
\begin{equation}\label[eq]{cover_prob}
  \Phi^{3^{+}}(\mathbf{X}_{\mathcal{N}}, \mathbf{y}) > 0,
\end{equation}
where $\Phi^{3^{+}}(\mathbf{X}_{\mathcal{N}}, \mathbf{y})$ will be referred to as the probability of coverage.

\subsection{Objective function derivation}\label[secc]{obj_formulation}
The objective function presented here is inspired by \cite{sun2014escaping}, but is modified for the purpose of multilateration. 
Here a stricter definition of coverage is applied (see \Cref{secc:coverage}).

In order to formulate a distributed optimization algorithm, the probability of coverage is rewritten with focus on a single drone $a$.
The results are stated in \eqref{stated_distr_cov_prob} and the derivations are available in Appendix \ref{appendix:derivations}.
\begin{subequations}\label[eq]{stated_distr_cov_prob}
  \begin{align}
    \Phi^{3^{+}}(\mathbf{X}_{\mathcal{N}}, \mathbf{y}) &= \Phi^{3^{+}}(\mathbf{X}_{\mathcal{N}\setminus\{a\}}, \mathbf{y}) + \Phi^{2}(\mathbf{X}_{\mathcal{N}\setminus\{a\}}, \mathbf{y})\hat{p}(\mathbf{x}_{a}, \mathbf{y})\label{eq:subeq1}\\
    &= (1-\hat{p}(\mathbf{x}_{a}, \mathbf{y}))\Phi^{3^{+}}(\mathbf{X}_{\mathcal{N}\setminus\{a\}}, \mathbf{y}) + \hat{p}(\mathbf{x}_{a}, \mathbf{y})\Phi^{2^{+}}(\mathbf{X}_{\mathcal{N}\setminus\{a\}}, \mathbf{y})\label{eq:subeq2}.
  \end{align}
\end{subequations}
From \eqref{eq:subeq2} it is clear that the probability of the point $\mathbf{y}$ being covered can be seen as an interpolation between two probability measures, 
with the local probability of agent $a$ as the interpolation variable. Low local probability of agent $a$ means that the probability of coverage supplied by the swarm to a larger extent depends on the coverage supplied by
the swarm excluding agent $a$. In the extreme case where the local probability of agent $a$ is zero, the probability of covering $\mathbf{y}$ depends only on the coverage supplied by the swarm excluding agent $a$.

Higher local probability of agent $a$ means that the contribution of $a$ towards covering the point $\mathbf{y}$ is greater. 
Thus less weight is put on the the probability of the swarm excluding agent $a$ covering the point. Instead more weight is put on the probability of at least \textit{two} other agents being able to communicate with an entity at $\mathbf{y}$. 
This is due to the fact that the probability of agent $a$ being able to communicate with said entity is higher, and only two or more agents must to be able to communicate with the entity at $\mathbf{y}$ to make the total number of agents covering $\mathbf{y}$ three or more.

\begin{figure}[H]
  \centering
  \includegraphics[width=.75\textwidth]{figs/local_objective_example.pdf}
  \caption{Non-zero regions of integrands in \eqref{rewritten_objective}. Note that perturbing the orange circle (position of agent $a$) does not affect the green region,
  as it is defined only by the intersections of the disks surrounding the blue points (other agents in the swarm).}
  \label[fig]{local_coverage_example}
\end{figure}

Using \eqref{eq:subeq1} the overall probability of coverage over the feasible space can be written as:
\begin{equation}\label[eq]{rewritten_objective}
  \begin{split}
    P(\mathbf{X}_{\mathcal{N}}) &=\int_{\mathcal{F}}\Phi^{3^{+}}(\mathbf{X}_{\mathcal{N}}, \mathbf{y})d\mathbf{y} =  \int_{\mathcal{F}}\Phi^{3^{+}}(\mathbf{X}_{\mathcal{N}\setminus\{a\}}, \mathbf{y}) + \Phi^{2}(\mathbf{X}_{\mathcal{N}\setminus\{a\}}, \mathbf{y})\hat{p}(\mathbf{x}_{a}, \mathbf{y})d\mathbf{y}\\
    &= \int_{\mathcal{F}}\Phi^{3^{+}}(\mathbf{X}_{\mathcal{N}\setminus\{a\}}, \mathbf{y})d\mathbf{y} + \int_{\mathcal{F}}\Phi^{2}(\mathbf{X}_{\mathcal{N}\setminus\{a\}}, \mathbf{y})\hat{p}(\mathbf{x}_{a}, \mathbf{y})d\mathbf{y}.
  \end{split}
\end{equation}
The first term in \eqref{rewritten_objective} is independent of the position of agent $a$ in both its domain and integrand. This independence is visualized in \figref{local_coverage_example}. 
Thus the overall coverage probability over $\mathcal{F}$ can be rewritten as:
\begin{equation}
  P(\mathbf{X}_{\mathcal{N}}) = P(\mathbf{X}_{\mathcal{N}\setminus\{a\}}) + P_{a}(\mathbf{X}_{\mathcal{N}}),
\end{equation}
where the \textit{local} probability of coverage for agent $a$ is defined as:
\begin{equation}\label[eq]{local_objective}
  P_{a}(\mathbf{X}_{\mathcal{N}}) = \int_{\mathcal{F}}\Phi^{2}(\mathbf{X}_{\mathcal{N}\setminus\{a\}}, \mathbf{y})\hat{p}(\mathbf{x}_{a}, \mathbf{y})d\mathbf{y}.
\end{equation}

As in \cite{sun2014escaping} it is noted that from the viewpoint of agent $a$, the swarm can be partitioned into three disjoint sets: $\{a\}$, $\mathcal{B}_{a}$ and $\mathcal{C}_{a}$.
Exploiting that all agents have homogenous range of communication, i.e.
\begin{equation}
  r_{a} = r\;\forall\:a\in\mathcal{N},
\end{equation}
the sets $\mathcal{B}_{a}$ and $\mathcal{C}_{a}$ are defined as:
\begin{subequations}\label[eq]{B_a_and_C_a_def}
  \begin{align}
    \mathcal{B}_{a} &= \{j\in\mathcal{N}\setminus\{a\}: \norm{\mathbf{x}_{a}-\mathbf{x}_{j}} \leq 2r\}\label[eq]{neigh_def}\\
    \mathcal{C}_{a} &= \{j\in\mathcal{N}\setminus\{a\}: \norm{\mathbf{x}_{a}-\mathbf{x}_{j}} > 2r\}.
  \end{align}
\end{subequations}
The set $\mathcal{B}_{a}$, from now on called the neighbors of $a$, contains all agents in the swarm, $\mathcal{N}$, whose communication disks form a non-empty intersection with that of $a$.
$\mathcal{C}_{a}$ contains all agents whose communication disks do not intersect with that of $a$. Furthermore note that
\begin{equation}
  \begin{split}
    \mathcal{B}_{a}\cap\mathcal{C}_{a} &= \emptyset\\
    \mathcal{B}_{a}\cup\mathcal{C}_{a}\cup\{a\} &= \mathcal{N}.
  \end{split}
\end{equation}

Applying \eqref{B_a_and_C_a_def} to \eqref{local_objective} yields:
\begin{equation}
  \begin{split}
    P_{a}(\mathbf{X}_{\mathcal{N}}) &= \int_{\mathcal{F}}\Phi^{2}(\mathbf{X}_{\mathcal{N}\setminus\{a\}}, \mathbf{y})\hat{p}(\mathbf{x}_{a}, \mathbf{y})d\mathbf{y}\\
    &= \int_{\mathcal{F}}\Big(\Phi^{2}(\mathbf{X}_{\mathcal{B}_{a}}, \mathbf{y}) + \Phi^{2}(\mathbf{X}_{\mathcal{C}_{a}}, \mathbf{y}) + \Phi^{1}(\mathbf{X}_{\mathcal{B}_{a}}, \mathbf{y})\Phi^{1}(\mathbf{X}_{\mathcal{C}_{a}}, \mathbf{y})\Big)\hat{p}(\mathbf{x}_{a}, \mathbf{y})d\mathbf{y}.
  \end{split}
\end{equation}
Partitioning the domain of integration into the visible set and invisible set of agent $a$, and noting that $\hat{p}(\mathbf{x}_{j}, \mathbf{y}) = 0\;\forall\;j\in\mathcal{C}_{a},\;\mathbf{y}\in V_{a}$ such that
$\Phi^{n}(\mathbf{X}_{\mathcal{C}_{a}}, \mathbf{y}) = 0\;\forall\;n\in\mathbb{Z}^{+},\;\mathbf{y}\in V_{a}$, and $\hat{p}(\mathbf{x}_{a}, \mathbf{y}) = 0\;\forall\;\mathbf{y}\in U_{a}$ yields:
\begin{equation}\label[eq]{local_objective_derivated}
  \begin{split}
    P_{a}(\mathbf{X}_{\mathcal{N}}) &= \int_{V_{a}}\Big(\Phi^{2}(\mathbf{X}_{\mathcal{B}_{a}}, \mathbf{y}) + \Phi^{2}(\mathbf{X}_{\mathcal{C}_{a}}, \mathbf{y}) + \Phi^{1}(\mathbf{X}_{\mathcal{B}_{a}}, \mathbf{y})\Phi^{1}(\mathbf{X}_{\mathcal{C}_{a}}, \mathbf{y})\Big)\hat{p}(\mathbf{x}_{a}, \mathbf{y})d\mathbf{y}\\
    &+ \int_{U_{a}}\Big(\Phi^{2}(\mathbf{X}_{\mathcal{B}_{a}}, \mathbf{y}) + \Phi^{2}(\mathbf{X}_{\mathcal{C}_{a}}, \mathbf{y}) + \Phi^{1}(\mathbf{X}_{\mathcal{B}_{a}}, \mathbf{y})\Phi^{1}(\mathbf{X}_{\mathcal{C}_{a}}, \mathbf{y})\Big)\hat{p}(\mathbf{x}_{a}, \mathbf{y})d\mathbf{y}\\
    &= \int_{V_{a}}\Phi^{2}(\mathbf{X}_{\mathcal{B}_{a}}, \mathbf{y})p(\norm{\mathbf{x}_{a}-\mathbf{y}})d\mathbf{y} = L(\mathbf{X}_{\mathcal{B}_{a}\cup\{a\}}).
  \end{split}
\end{equation}
Thus the local probability of coverage for an agent $a$ is dependent on the position, $\mathbf{x}_{a}$, of agent $a$ in both domain and integrand, and the positions of the neighbors of agent $a$.
The set of agents consisting of agent $a$ and its neighbors will be referred to as agent $a$'s local swarm.\clearpage

As discussed in subsection \ref{trilat} it is beneficial that agents used for multilateration spread out to some extent in order to ensure sufficient accuracy of multilateration.
Furthermore ensuring that agents spread throughout the mission space is desirable. 
In order to explicitly encourage spread of agents, a term penalizing an agent $a$ for being close to another agent $j$ is introduced. 
When agents are close to each other this term should induce some velocity in the agents causing
them to spread.
In \cite{pot_field} virtual potential fields are used to disperse agents. There the potential between an agent $a$ and its neighbor $j$ is modelled as:
\begin{equation}\label[eq]{pt_field}
  U = -k\frac{1}{\norm{\mathbf{x}_{a} - \mathbf{x}_{j}}},
\end{equation}
where $k$ is a constant defining the strength of the field.
Drawing inspiration from this the proximity term between two agents, $a$ and $j$, is defined according to:
\begin{equation}\label[eq]{proximity_func}
  D(\mathbf{x}_{a}, \mathbf{x}_{j}) = k_{1}e^{-k_{2}\norm{\mathbf{x}_{a} - \mathbf{x}_{j}}}.
\end{equation}
As opposed to \eqref{pt_field}, \eqref{proximity_func} is defined for all $\mathbf{x}_{a},\:\mathbf{x}_{j}$ allowing for easier implementation. The function has two tunable parameters, $k_{1},\:k_{2}\geq 0$, whose effects are show in \figref{cdr}.

Using \eqref{local_objective_derivated} and \eqref{proximity_func} the \textit{local} objective function for an agent $a$ is defined as:
\begin{equation}\label[eq]{local_objective_func}
  H(\mathbf{X}_{\mathcal{B}_{a}\cup\{a\}})  = L(\mathbf{X}_{\mathcal{B}_{a}\cup\{a\}})  - \sum_{j\in\mathcal{B}_{a}}D(\mathbf{x}_{a}, \mathbf{x}_{j}).
\end{equation}
Due to the negative sign in front of the sum of proximity functions this will be referred to this as the active dispersion term.
\begin{figure}[H]
  \centering
  \includegraphics[width=.7\textwidth]{figs/close_dist_repell_example.pdf}
  \caption{Proximity term in \eqref{proximity_func} for an agent, $a$, and its neighbor, $j$.}
  \label[fig]{cdr}
\end{figure}

\subsection{Constraints}\label[secc]{constraints}
It is clear from subsection \ref{trilat} that agents used as beacons in a multilateration scheme cannot be positioned on a straight line, as this would make it impossible
to uniquely determine the unknown position of an entity. Only agents close to an entity whose position is to be determined will
take part in the multilateration scheme. Due to this it is imposed that only agents that together can be used for multilateration must satisfy the non-linear position requirement.

For an agent $a$ with $N$ neighbors $j\in\mathcal{B}_{a},\:|B_{a}|=N$ the matrix $\mathbf{V}$ is constructed according to:
\begin{equation}\label[eq]{V_def}
  \mathbf{V}(\mathbf{X}_{\mathcal{B}_{a}}) = \begin{bmatrix}
    \mathbf{X}_{\mathcal{B}_{a, 0}}-\mathbf{X}_{\mathcal{B}_{a, 1}}&\hdots&\mathbf{X}_{\mathcal{B}_{a, 0}}-\mathbf{X}_{\mathcal{B}_{a, N-1}}
  \end{bmatrix}.
\end{equation}
If agent $a$ has less than two neighbors, or all of agent $a$'s neighbors are positioned on a straight line, the matrix $\mathbf{V}(\mathbf{X}_{\mathcal{B}_{a}})$ will not span $\mathbb{R}^{2}$, i.e. $\mathrm{Rank}(\mathbf{V}(\mathbf{X}_{\mathcal{B}_{a}})) < 2$.

Assuming that agent $a$ has two or more neighbors, $|\mathcal{B}_{a}|\geq 2$, and all neighbors of $a$ are positioned on a straight line, it is desirable that agent $a$ is positioned sufficiently far away from the line. This constraint is implemented as follows:
Without loss of generality, define:
\begin{subequations}\label[eq]{vl}
  \begin{align}
    \mathbf{v}(\mathbf{x}_{a}) &= \mathbf{x}_{a} - \mathbf{X}_{\mathcal{B}_{a}, 0}\\
    \mathbf{l} &= \mathbf{X}_{\mathcal{B}_{a}, 1} - \mathbf{X}_{\mathcal{B}_{a}, 0},
  \end{align}
\end{subequations}
meaning $\mathbf{v}(\mathbf{x}_{a})$ is the vector from agent $a$ to its first neighbor, and $\mathbf{l}$ is the vector between agent $a$'s first and second neighbor. Thus $\mathbf{l}$ is parallel to the line 
that goes through all of agent $a$'s neighbors.
Equation (3.98) in \cite{projection} is used to project $\mathbf{v}(\mathbf{x}_{a})$ into $\mathbf{l}$. Thus the component of $\mathbf{v}(\mathbf{x}_{a})$ that is parallel to the line
through agent $a$'s neighbor $i$ and $j$ is obtained as:
\begin{equation}
  \mathbf{v}_{\parallel}(\mathbf{x}_{a}) = \frac{\mathbf{v}(\mathbf{x}_{a})^{T}\mathbf{l}}{\mathbf{l}^{T}\mathbf{l}}\mathbf{l}.
\end{equation}
The component of $\mathbf{v}(\mathbf{x}_{a})$ perpendicular to the line through agent $a$'s
neighbors is obtained as:
\begin{equation}
  \mathbf{v}_{\perp}(\mathbf{x}_{a}) = \mathbf{v}(\mathbf{x}_{a}) - \mathbf{v}_{\parallel}(\mathbf{x}_{a}).
\end{equation}
Now the distance from agent $a$ to the line through its neighbors can be computed as:
\begin{equation}
  \norm{\mathbf{v}_{\perp}(\mathbf{x}_{a})} = \norm{\mathbf{v}(\mathbf{x}_{a}) - \frac{\mathbf{v}(\mathbf{x}_{a})^{T}\mathbf{l}}{\mathbf{l}^{T}\mathbf{l}}\mathbf{l}}.
\end{equation}
Seen as it is impossible to place an agent such that it lays on a straight line through all it's neighbors if the neighbors do not already
lay on a straight line, it is demanded that the non-linear position constraint must be fulfilled only when $\mathrm{Rank}(\mathbf{V}(\mathbf{X}_{\mathcal{B}_{a})}) < 2$, where $\mathbf{V}(\cdot)$ is defined in \eqref{V_def}. The non-linear position constraint is defined as:
\begin{equation}\label[eq]{non_lin_pos}
    \norm{\mathbf{v}(\mathbf{x}_{a}) - \frac{\mathbf{v}(\mathbf{x}_{a})^{T}\mathbf{l}}{\mathbf{l}^{T}\mathbf{l}}\mathbf{l}}\geq d_{min},
\end{equation}
where $d_{min}$ is a tunable parameter that defines how close an agent is allowed get to the line through it's neighbors, and $\mathbf{v}$ and $\mathbf{l}$
are defined in \eqref{vl}.

Furthermore it is demanded that any two agents must be some minimum distance apart at any given time. This is due to the fact that agents colliding could cause damage to the hardware, and possibly render them unusable.
This constraint is modelled as:
\begin{equation}
  \norm{\mathbf{x}_{a} - \mathbf{x}_{j}} \geq r_{min} \;\forall\;j\in\mathcal{B}_{a},
\end{equation}
where $r_{min}$ is a tunable parameter that sets a limit to how close an agent can be positioned to any of its neighbors.\clearpage
\subsection{Optimization problem formulation}
The non-linear position constraint \eqref{non_lin_pos} presented in \ref{constraints} is dependent on the neighbor set of agent $a$, $\mathcal{B}_{a}$, as it should only be imposed if the 
neighbors of agent $a$ lay on a straight line. Using the objective function in \eqref{local_objective_func} and the constraints in discussed in \ref{constraints}, the local optimization problem for an agent $a$ is defined as:
\begin{subequations}\label[eq]{local_opt_prob}
  \begin{align}
    \begin{split}\label[eq]{totally_objective}
      &\max_{\mathbf{x}_{a}}\;H(\mathbf{X}_{\mathcal{B}_{a}\cup\{a\}})\\
    \end{split}\\
    \mathrm{s.t.}\;
    \begin{split}
      &\mathbf{x}_{a}\in\mathcal{F}
    \end{split}\\
    \begin{split}\label[eq]{min_two_neig}
      &|\{j\in\mathcal{B}_{a}: \norm{\mathbf{x}_{a} - \mathbf{x}_{j}} \leq 2r\}|\geq 2
    \end{split}\\
    \begin{split}\label[eq]{min_dist_neigh}
      &\norm{\mathbf{x}_{a} - \mathbf{x}_{j}} \geq r_{min}\;\forall\;j\in\mathcal{B}_{a}
    \end{split}\\
    \begin{split}\label[eq]{non_linear_neighb}
      &\norm{\mathbf{v}(\mathbf{x}_{a}) - \frac{\mathbf{v}(\mathbf{x}_{a})^{T}\mathbf{l}}{\mathbf{l}^{T}\mathbf{l}}\mathbf{l}}\geq d_{min},\quad\mathrm{iff.}\;\mathrm{Rank}(\mathbf{V}(\mathbf{X}_{\mathcal{B}_{a}}))<2
    \end{split}
\end{align}
\end{subequations}
where $\mathbf{V}(\cdot)$ is defined in \eqref{V_def} and $\mathbf{v}(\cdot)$ and $\mathbf{l}$ are defined in \eqref{vl}.

In situations where the local probability of coverage of agent $a$ is constant, i.e. $\frac{\partial L(\mathbf{X}_{\mathcal{B}_{a}\cup\{a\}})}{\partial \mathbf{x}_{a}} = \mathbf{0}$, the solution of \eqref{totally_objective} will be achieved at the position $\mathbf{x}_{a}$ that minimizes the dispersion term in \eqref{local_objective_func}.
This will cause agent $a$ to move as far away from its neighbors as possible. Such behavior is undesirable as this might render agent $a$ neighbor-less. An agent 
$a$ without neighbors fulfills $\mathcal{B}_{a} = \emptyset$ and hence $H(\mathbf{X}_{\mathcal{B}_{a}\cup\{a\}}) \equiv 0$. Thus re-optimization will result in $\mathbf{x}_{a}$ being unaltered, causing agent $a$
to stay at $\mathbf{x}_{a}$ indefinitely. To prevent such behavior the constraint \eqref{min_two_neig} is imposed. This will prevent agent $a$ from completely disconnecting from its local swarm. 

Furthermore, observe that Rank$(\mathbf{V}(\mathbf{X}_{\mathcal{B}_{a}}))<2$ if either all neighbors of agent $a$ lie on a straight line or agent $a$ has less than two neighbors. If agent $a$ has less than
two neighbors it is not possible to draw a line between the neighbors of agent $a$, and thus the non-linear neighbor constraint \eqref{non_linear_neighb} is undefined. This further motivates imposing the constraint that
agent $a$ should have at least two neighbors at the solution to \eqref{local_opt_prob}.

The optimization problem formulated in \eqref{local_opt_prob} is non-convex \cite{NoceWrig06_convex_prob}. The objective function is generally non-concave and the constraints form a non-convex feasible set. In 
\figref{objective_example} a visualization of the objective function is shown for an agent with 5 neighbors.
\begin{figure}[H]
  \centering
  \includegraphics[width = .8\textwidth]{figs/objective_example.pdf}
  \caption{Objective function in \eqref{totally_objective} for agent $a$ with $|B_{a}| = 5$.}
  \label[fig]{objective_example}
\end{figure}