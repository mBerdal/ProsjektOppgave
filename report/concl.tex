\section{Conclusion}
In this project, a novel distributed probability-based coverage optimization algorithm is proposed. The objective of which
is to, without centralized control, form a network of agents who together can be used as beacons in a 
multilateration scheme allowing for accurate localization of entities entering the environment in which the agents are located.

The distributed objective function consists of a measure of local probability of coverage and a potential field-based measure
of dispersion. The dispersion term is added so that agents are induced to spread from other agents.
The algorithm is tested in four environments of different sizes and complexity. In each environment swarms of different sizes
are deployed, initiated in a dense configuration, and the algorithm is applied with and without actively dispersing agents.

Simulations show that when the proposed algorithm is run without actively dispersing agents, the swarm of agents
is prone to converging to configurations in which only a small part of the swarm explores the environment, leading to 
clearly sub-optimal coverage. 
This is due to the \textit{local} probability of coverage being a constant function of an agents position in regions covered by four or more agents, 
leading to zero step size when performing local optimization.
Applying active dispersion mitigates the problem of agents not exploring the environment. This, in some cases, results in far better 
coverage as the swarm is able to escape the initial dense configuration.
However, in environments of smaller sizes simulations show that actively dispersing agents can overshadow the importance of covering the area, thus 
leading to sub-optimal configurations in which maximum dispersion is obtained rather than maximum coverage.

The results presented in this report show that in all tested environments, for all swarm sizes, the proposed algorithm yields
a configuration in which the covered area is greater than or equal to the initially covered area when only optimizing the local probability of coverage. 
Furthermore, it is shown that in some cases adding a dispersion term inducing agents to explore the environment can drastically improve the coverage supplied
by the swarm of agents. However, seen as the algorithm lacks robustness it can not yet be utilized in real-world situations. Work on the deployment problem
will continue in an upcoming master's thesis. There, the problems presented in this report will be addressed, and hopefully, result in a solution that can
be utilized during search-and-rescue missions.